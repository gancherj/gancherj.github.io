\documentclass{article}

\usepackage{geometry}
\geometry{a4paper,margin=1cm, bottom=0.5cm}
\newcommand{\s}{\mathsf{h}}
\newcommand{\irule}{\mathsf{h}}
\setlength\parindent{0pt}
\begin{document}
\begin{center}
    \Huge Joshua Gancher
\end{center}
\begin{center}
    gancher.dev $\mid$ jgancher@andrew.cmu.edu $\mid$ (503) 341-1176 $\mid$ 5923
    Nicholson St, Pittsburgh PA 15217
 \end{center}

{\huge Research Interests}
\vspace{3pt}
 \hrule
\vspace{5pt}
I apply tools from Formal Methods and Programming Languages 
to construct,
certify, and give formal semantics to secure systems.
I am particularly interested in reasoning about computational security for cryptographic mechanisms in practice.
Interests: applied cryptography, security protocols, type systems, compiler
correctness, static analysis, interactive proof assistants, program logic, distributed systems

\vspace{10pt}
{\huge Education}
\vspace{3pt}
 \hrule
\begin{itemize}
    \item {\bf Ph.D. in Computer Science}. Cornell University. December 2021.
        \begin{itemize}
            \item Advised by Elaine Shi. Thesis:  Equational Reasoning for Verified Cryptographic
                Security.
        \end{itemize}
    \item {\bf B.A. in Mathematics}. Reed College. May 2016.
        \begin{itemize}
            \item Thesis: Fully Homomorphic Encryption.
        \end{itemize}
\end{itemize}

\vspace{10pt}
{\huge Experience and Appointments}
\vspace{3pt}
 \hrule
\begin{itemize}
    \item {\bf Postdoctoral Fellow}. Carnegie Mellon University. 2021 - Present.
        \begin{itemize}
            \item Advised by Bryan Parno. Research Focus: Type systems for secure cryptographic protocols. 
        \end{itemize}

    \item {\bf Amazon Automated Reasoning Group}. Software Engineering Intern.
        Summer 2019.
        \begin{itemize}
            \item Delivered formal proofs and specifications for Amazon Encryption SDK
            \item Created a compiler from internal protocol description language to Dafny
        \end{itemize}

    \item {\bf Galois, Inc.} Software Engineering/Research Intern. Summer 2017.
        \begin{itemize}
            \item Worked with Air Force Research Lab to migrate codebase to Rust
            \item Extended Crucible symbolic execution engine to handle Rust
        \end{itemize}

\end{itemize}
    {\bf Programming Experience:} Haskell, Coq, Dafny, F$^*$, OCaml, Rust, C/C++
    \\
    {\bf Professional Activities:} Program Committees: FCS 2020, FCS 2023; External Reviewer: CCS 2017, CCS 2021
    \\
    {\bf Selected Talks:} 
    \begin{itemize}
        \item Boston POPV Seminar, 2023: Owl: Compositional Verification of
            Security Protocols via an Information-Flow Type System
        \item New England Systems Verification Day, 2022:  End-to-End Verification for Security Protocols
        \item Stanford Research Lunch, 2022: A Core Calculus for Equational Proofs of Cryptographic Protocols
        \item PLCrypt Workshop, 2022: End-to-End Verification for Security Protocols in F* 
        \item New England Systems Verification Day, 2019: IPDL: Proving Compositional Security of Cryptographic Protocols
    \end{itemize}

\vspace{10pt}
{\huge Publications}
\vspace{3pt}
 \hrule
\begin{itemize}
    \item {\bf OWL: Compositional Verification of Security Protocols
        via an Information-Flow Type System}. \\ IEEE S\&P 2023 \\
    Joshua Gancher, Sydney Gibson, Pratap Singh, Samvid Dharanikota, and Bryan
        Parno.
    \item {\bf A Core Calculus for Equational Proofs of Cryptographic
        Protocols}. POPL 2023. \\ Joshua Gancher, Kristina Sojakova, Xiong Fan,
        Elaine Shi, and Greg Morrisett.  
    \item {\bf Viaduct: An Extensible, Optimizing Compiler for Secure
        Distributed Programs}. PLDI 2021.
        \\
        Coşku Acay, Rolph Recto, Joshua Gancher, Andrew Myers, and Elaine Shi.
    \item \textbf{Symbolic Proofs for Lattice-Based Cryptography}. CCS 2018. 
        \\
    {Gilles Barthe, Xiong Fan, Joshua Gancher, Benjamin Grégoire, Charlie Jacomme and Elaine Shi.}
    \item \textbf{Externally Verifiable Oblivious RAM}. PETS 2017.
        \\
    Joshua Gancher, Adam Groce, and Alex Ledger.
\end{itemize}

\vspace{10pt}
{\huge Funding}
\vspace{3pt}
 \hrule
 \begin{itemize}
     \item {\bf NSF: SatC: CORE: Small: Automating the End-to-End Verification
         of Security Protocol Implementations.} 2022.
         \\
         Award \# 2224279. Award size: \$600,000. PIs: Bryan Parno and Joshua
         Gancher.
         \\
         Advancing the state of the art in modular, highly automated, end-to-end formal
         proofs for security protocols. 
 \end{itemize}

 % Program Committee: FCS 2020; external reviewer for CCS 2021; CCS 2017


 % Talks: NESVD 2022: ; PLCrypt; Stanford Research L; 
% PLCrypt Workshop:  End-to-End Verification for Security Protocols in F*;
 % 5/22/2022
% Stanford Research Lunch: A Core Calculus for Equational Proofs of
 % Cryptographic Protocols; 11/10/22
% NESVD 2022: End-to-End Verification for Security Protocols
% NESVD 2019: IPDL: Proving Compositional Security of Cryptographic Protocols

\end{document}

\documentclass{article}

\usepackage{geometry} 
\usepackage{hyperref}
\usepackage{enumitem}
\geometry{a4paper,margin=2cm, bottom=3cm}
\newcommand{\s}{\mathsf{h}}
\newcommand{\irule}{\mathsf{h}}
\newcommand{\rsection}[1]{\vspace{1em} {\huge #1} \vspace{0.5em} \hrule \vspace{1em}}
\setlength\parindent{0pt}
\begin{document}
\begin{center}
    \Huge Joshua Gancher
\end{center}
\begin{center}
    \url{gancher.dev} $\mid$ jgancher@andrew.cmu.edu $\mid$ (503) 341-1176 $\mid$ 5923
    Nicholson St, Pittsburgh PA 15217
 \end{center}

\rsection{Research Interests}

I apply tools from Formal Methods and Programming Languages 
to construct,
certify, and give formal semantics to secure systems.
I am particularly interested in reasoning about security for cryptographic
mechanisms used in practice.
Broadly, I am interested in applied cryptography, distributed systems, type systems, compiler
correctness, proof assistants, and formal methods. I have published in
\textbf{IEEE S\&P, POPL, CCS, PLDI,} and \textbf{PETS}. 

\rsection{Education}
\begin{itemize}
    \item {\bf Ph.D. in Computer Science}. Cornell University. December 2021.
        \begin{itemize}
            \item Co-advised by Elaine Shi and Greg Morrisett. Thesis:  Equational Reasoning for Verified Cryptographic
                Security.
        \end{itemize}
    \item {\bf B.A. in Mathematics}. Reed College. May 2016.
        \begin{itemize}
            \item Thesis: Fully Homomorphic Encryption.
        \end{itemize}
\end{itemize}

\rsection{Experience and Appointments}
\begin{itemize}
    \item {\bf Postdoctoral Fellow}. Carnegie Mellon University. 2021 - Present.
        \begin{itemize}
            \item Advised by Bryan Parno. Research Focus: Type systems for secure cryptographic protocols. 
        \end{itemize}

    \item {\bf Amazon Automated Reasoning Group}. Software Engineering Intern.
        Summer 2019.
        \begin{itemize}
            \item Delivered formal proofs and specifications for Amazon Encryption SDK
            \item Created a compiler from internal protocol description language to Dafny
        \end{itemize}

    \item {\bf Galois, Inc.} Software Engineering/Research Intern. Summer 2017.
        \begin{itemize}
            \item Worked with Air Force Research Lab to migrate codebase to Rust
            \item Extended Crucible symbolic execution engine to handle Rust
        \end{itemize}
\end{itemize}
    % {\bf Programming Experience:} Haskell, Coq, Dafny, F$^*$, OCaml, Rust, C/C++
    % \\
    {\bf Professional Activities:} Program Committees: FCS 2020, FC 2023, SPLASH
    SRC 2023; 

    External/Shadow Reviewer for CCS 2017, CSF 2020, CCS 2021, POPL 2024
    %\\
    %{\bf Selected Talks:} 
    %\begin{itemize}
    %    \item New England Systems Verification Day, 2022:  End-to-End Verification for Security Protocols
    %    \item Stanford Research Lunch, 2022: A Core Calculus for Equational Proofs of Cryptographic Protocols
    %    \item PLCrypt Workshop, 2022: End-to-End Verification for Security Protocols in F* 
    %    \item New England Systems Verification Day, 2019: IPDL: Proving Compositional Security of Cryptographic Protocols
    %\end{itemize}
    \\
    \\
    {\bf Teaching:} Reed College Thesis Advisor, 2022-2023; TA for CS 3410
    (Computer System Organization and Programming); TA for CS 4120 (Introduction
    to Compilers)
    \\
    \\
    {\bf Professional Service:} PhD Admissions Volunteer for Cornell, 2019

\rsection{Publications}
\begin{itemize}[leftmargin=*]
    \item {\bf Secure Synthesis of Distributed Cryptographic Applications}. \\
        In submission to CSF 2024. \\ 
        Cosku Acay, Joshua Gancher, Rolph Recto, and Andrew Myers.
    \item {\bf OWL: Compositional Verification of Security Protocols
        via an Information-Flow Type System}. \\ IEEE S\&P 2023. \\
    Joshua Gancher, Sydney Gibson, Pratap Singh, Samvid Dharanikota, and Bryan
        Parno.
    \item {\bf A Core Calculus for Equational Proofs of Cryptographic
        Protocols}. \\ POPL 2023. \\ Joshua Gancher, Kristina Sojakova, Xiong Fan,
        Elaine Shi, and Greg Morrisett.  
    \item {\bf Viaduct: An Extensible, Optimizing Compiler for Secure
        Distributed Programs}. \\ PLDI 2021.
        \\
        Coşku Acay, Rolph Recto, Joshua Gancher, Andrew Myers, and Elaine Shi.
    \item \textbf{Symbolic Proofs for Lattice-Based Cryptography}. \\ CCS 2018. 
        \\
    {Gilles Barthe, Xiong Fan, Joshua Gancher, Benjamin Grégoire, Charlie Jacomme and Elaine Shi.}
    \item \textbf{Externally Verifiable Oblivious RAM}. \\ PETS 2017.
        \\
    Joshua Gancher, Adam Groce, and Alex Ledger.
\end{itemize}

\rsection{Funding}
 \begin{itemize}[leftmargin=*]
     \item {\bf NSF: SatC: CORE: Small: Automating the End-to-End Verification
         of Security Protocol Implementations.} 2022.
         \\
         Award \# 2224279. Award size: \$600,000. PIs: Bryan Parno and Joshua
         Gancher.
         \\
         Advancing the state of the art in modular, highly automated, end-to-end formal
         proofs for security protocols. 
 \end{itemize}

%\rsection{References}
%\begin{itemize}
%\item Bryan Parno: parno@cmu.edu
%\item Elaine Shi: runting@gmail.com 
%\item Greg Morrisett: jgm19@cornell.edu
%\item Andrew Myers: andru@cs.cornell.edu
%\end{itemize}

\rsection{Invited Talks}
\begin{itemize}
    \item IETF 118, November 2023: Owl: New Directions for Security Protocol Analysis
    \item CyLab Partners Conference 2023: Verifying Security Protocols
        End-to-End with Owl
    \item CMU Crypto Seminar, September 2023: Owl: Compositional Verification of Security Protocols
    \item CMU PoP Seminar, September 2023: Owl: Compositional Verification of Security Protocols 
    \item INRIA Prosecco Seminar, June 2023: Owl: Compositional Verification of Security Protocols
    \item Boston University POPV Seminar, April 2023: Owl: Compositional Verification of Security Protocols via an Information-Flow Type System
    \item Galois Tech Talk, March 2023: End-to-End Verification for Security Protocols
    \item Stanford Software Research Lunch, November 2022: A Core Calculus for Equational Proofs of Cryptographic Protocols
    \item New England Systems Verification Day 2022: End-to-End Verification for Security Protocols
    \item PLCrypt Workshop, May 2022: End-to-End Verification for Security Protocols in F*
    \item New England Systems Verification Day 2019: IPDL: Proving Compositional Security of Cryptographic Protocols
\end{itemize}

\end{document}
